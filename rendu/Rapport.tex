\documentclass[11pt]{article}

\usepackage[utf8]{inputenc}
\usepackage[T1]{fontenc}
\usepackage[francais]{babel}
\usepackage{fancyhdr}
\usepackage[scale=0.7]{geometry}
\usepackage{listings}
\usepackage{color}
\usepackage{listingsutf8}
\usepackage{enumerate}
\usepackage{hyperref}

\pagestyle{fancy}
\renewcommand{\footrulewidth}{1pt}
\fancyfoot[C]{\textbf{page \thepage}}
\fancyfoot[L]{Polytech Nantes}
\fancyfoot[R]{2015 -- 2016}
\fancyhead[L]{Hugo \bsc{Pigeon}, Tudal \bsc{Lebot}, Pierre \bsc{Pétillon}, Ronan \bsc{Jamet}}
\fancyhead[R]{INFO 3 Groupe 3}

\title{Manuel de PolyBigBalance}
\author{Hugo \bsc{Pigeon}, Tudal \bsc{Lebot}, Pierre \bsc{Pétillon},Ronan \bsc{Jamet}}

\definecolor{mygray}{rgb}{0.5, 0.5, 0.5}
\definecolor{cyan}{rgb}{0.0, 0.6, 0.6}
\definecolor{maroon}{rgb}{0.5,0,0}
\definecolor{darkgreen}{rgb}{0,0.5,0}

\lstdefinelanguage{BASH}
{
  basicstyle=\ttfamily\footnotesize,
  morestring=[b]",
  moredelim=[s][\bfseries\color{maroon}]{<}{\ },
  moredelim=[s][\bfseries\color{maroon}]{</}{>},
  moredelim=[l][\bfseries\color{maroon}]{/>},
  moredelim=[l][\bfseries\color{maroon}]{>},
  morecomment=[s]{<?}{?>},
  morecomment=[s]{<!--}{-->},
  commentstyle=\color{darkgreen},
  stringstyle=\color{blue},
  identifierstyle=\color{red}
}

\lstset{
	inputencoding=utf8/latin1,
	language=BASH,
	columns=flexible,
	showstringspaces=false,
	basicstyle=\ttfamily,
	keywordstyle=\color{cyan},
	commentstyle=\color{mygray},
	breaklines,
	breakindent=1.5em,
	xleftmargin=2em,
	xrightmargin=2em,
	frame=single,
	numbers=left,
	linewidth=15.5cm
}

\makeatletter
\let\ps@plain\ps@fancy 
\makeatother

\begin{document}

	\maketitle

	\section{Téléchargement}
		
		Le code source du projet est disponible à cette adresse: \href{https://gitlab.univ-nantes.fr/E133567G/poly-big-balance/tree/master/}{polybigalance}\\
		\\
		Des librairies indépendantes ont été créees pour le jeu, voici les liens vers leur code source:\\
		\href{https://gitlab.univ-nantes.fr/ronan/coraMaths/tree/master}{Librairie Mathématique}\\
		\href{https://gitlab.univ-nantes.fr/ronan/coraGraphics/tree/master}{Librairie Graphique}\\
		\href{https://gitlab.univ-nantes.fr/ronan/coraPhysics/tree/master}{Librairie Physique}\\
	

	\section{Installation}
	
		Pour compiler et créer un jar depuis le code source, il suffit de lancer à la racine du code la commande:
		\verb|mvn install|
	
	\section{Lancement}
		Pour lancer le jeu, il existe deux solutions
		
		\subsection{Maven}
		Depuis maven, la commande suivante est à réaliser:
		
		\verb|mvn exec:exec|
		
		\subsection{Jar}
		Vous pouvez aussi lancer le jeu en lançant un jar issu de la compilation maven\\
		\verb|java -jar polybigbalance-1.0-SNAPSHOT.jar|\\
		Le jar peut être directement récupéré depuis le dépos gitlab grâce au lien suivant: 
		\href{https://gitlab.univ-nantes.fr/E133567G/poly-big-balance/tree/master/polybigbalance/polybigbalance-1.0/SNAPSHOT}{Version-1.0}.\\
		D'autres versions seront disponibles avec quelques améliorations à cette adresse:\\ \href{https://gitlab.univ-nantes.fr/E133567G/poly-big-balance/tree/master/polybigbalance/}{Nouvelles Versions}.\\
		
		\subsection{Windows}
		Pour windows, le plus simple est de créer un fichier run.bat contenant la commande: \verb|java -jar polybigbalance-1.0-SNAPSHOT.jar|\\
		Il suffit ensuite de lancer le bat.
		
\end{document}
